% !TeX root = ../main.tex

\chapter{结论}
\label{cha:chapter99}

本文使用事件巧妙地定义了一些典型的社交网络图结构的时序动态性,用一些结构特征的变化来模拟真实社交网络中的变化特性,是一个新颖的动态社交网络图定义方式。这一定义方法高度抽象化,聚焦于图结构本身的特征,并非完全使用模拟的方式进行动态的定义,因此可以方便、快速地通过这种定义进行动态图的生成。

基于对动态图的认识与研究,本文在已有的静态社交网络图快速生成方法\cite{FastSGG}基础上进行了改进和扩展,成功将这篇工作中提到的图生成方法扩展应用于动态图的生成过程,提出并实现了一个动态图的配置与生成算法,并且构建了一个完整的动态社交网络图生成系统。

在本文设计与实现的动态图生成算法中,设计并应用了节点映射(NodeMap)的概念,将许多动态图的事件定义关联到这一结构中,从而顺利进行动态图的事件执行。

% 同时,文中进行了大量的优化方面的尝试,成功提速了动态图的生成工作,节省了效率。

本文实现的系统中使用前后端架构,利用Vue.js、Echarts、ElementUI进行前端配置、可视化展示体系的搭建,利用Django进行后端资源、任务的管理,并使用多线程技术进行生成过程的管理,有效提升了系统的可用性、易用性。与原始的JSON形式配置方法相比,文中提供的动态图生成管理系统可以更高效、更方便地进行动态图的配置、运行、结果查看与可视化,并且可以方便用户进行远程生成与管理。

% 在这次课题的研究过程中,笔者对动态图的特性以及社交网络的一些特征有了更深入的理解,此次项目也锻炼了笔者的前后端技术应用水平和软件工程开发水平,较全面地提升了笔者的专业素养。

本次研究主要遇到的问题就是如何定义与实现动态社交网络图的动态性、如何在邻接表上用随机方式进行高效生成而非直接使用模拟的方式生成。本文关注社交网络本身的一些结构特征信息,包括图中节点度数的分布情况、节点的度数相对排序,因此使用了名为节点映射(NodeMap)的结构,将节点按照其度数大小进行排序组织,让节点蕴含的度数特性在节点映射中得以保存。在相关事件的处理过程中就可以通过对节点映射的修改来实现。

除此之外,在动态图管理生成系统的搭建过程中也曾经遇到一些问题,如怎样实现数据项的动态添加、如何更高效地进行图的可视化、如何避免耗时较长的生成过程阻塞后端服务进程等。为了解决前端配置与可视化展示的相关问题,本文中使用了流行的Vue.js框架与echarts框架进行数据的绑定与可视化,用ElementUI进行界面的美化操作。为了解决后端阻塞的问题,文中使用多线程的方式进行处理,收到生成请求之后在后台开启一个新的线程,线程结束后将结果反馈给主线程以便用户进行结果的查看、分析。

由于时间与经验所限,本课题中提出的用事件进行定义、用节点映射(NodeMap)进行实现的方法并不是尽善尽美,这样的方式使用的是图结构的相关特征,关注点在于动态图的结构方面的变化,而一定程度上忽视了结构变化内在的联系与规律,如真实网络中一些节点的变化趋势并非相同,符合某些特征的节点更有可能在给定的事件中受到影响,实践中相关数据生成、事件处理方面使用的随机方式有很大的提升空间。并且本次研究中高度自由化的配置方法使得用户在所提供的可配置选项中进行的配置不一定可以符合真实网络中的相关特征,一些动态图的内部节点间关联等更多特征难以用这里提出的基于随机的动态图模拟方法进行模拟。

对于动态社交网络图的研究还有很大的空间,在对真实网络进行后续分析的基础上可能可以提出更合适的方法。结合神经网络的方式,用RNN提取之前的节点特征来进行后续事件配置,指导事件中影响到的节点的选择与具体的事件执行过程,可能会有更符合真实情况的结果。