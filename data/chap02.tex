% !TeX root = ../main.tex

\chapter{动态图特性分析与生成器配置方式研究}
\label{cha:chapter02}

本章前两节将从图的节点和边、图的结构特征入手,对这两方面的动态性进行分析,进而提出用“事件”进行图结构特征动态性表现的方法。在第三节中,将提出一个包含动态特征的图生成器配置方式。

本章中提到的“动态图”是指具有时间序列结构的社交网络图,是普通图在时间轴上的扩展,可以展示图的动态变化过程。因此,动态社交网络图相比于静态的社交网络图,蕴含着更多的信息,能够更好地体现社交网络中的一些结构特征、演化特点,能够让数据更好地为人们所用。

例如:

\begin{itemize}
    \item 在电商平台的推荐算法中,可以用一个二分图来表示用户和商品之间的关注、购买关系,从而利用这些数据进行推荐\cite{7009419}。由于用户的需求往往具有时效性,可能在某一段时间内关注一类商品(如在新房装修时会搜索有关家具、电器的内容)但是这段时间之后就不会有类似的需求。如果在这样的数据图中加入时间信息,就可以更好地利用在某一个时间窗口中用户与商品之间、商品与商品之间的相关性,做出更好的推荐,如一篇对推荐系统进行研究的文章\cite{NAJAFABADI2019526}中就考虑了消费在时间上的重叠性。
    \item 在反洗钱的操作中,可以将不同账户之间的转账关系建模成一个图结构,由于洗钱操作需要防侦查操作,往往是在很短的时间内进行了大笔金钱交易。在银行账户转账记录的图中加入时间信息可以更好地进行此类的分析操作\cite{pareja2019evolvegcn}。
\end{itemize}

\vspace{0.2cm}

和静态图相比,动态图最重要的特征就是其中的动态性。对动态图进行分析,首先需要对图的动态性进行建模,用一个更为具体的方式对这种性质进行描述。因此本章接下来将从动态性的分析、建模等角度进行介绍。

\section{节点与边上的动态性}
\label{cha:node_edge_dynamic}

所谓动态性,就是在时间序列上的变化。要探究图上动态性的体现方式,就需要先了解一个静态图的基本组成要件:

\begin{itemize}
    \item 节点:某个实体/事物的抽象化表示,可以具有一些属性值,属性可以用key-value对的字典形式表示;
    \item 边:节点之间的连接方式,可以有多个属性值。边可以按照以下几种不同的方式进行分类:
    \begin{itemize}
        \item 有向边与无向边
        \item 单边与多重边
        \item 带权重的边与无权重的边
    \end{itemize}
\end{itemize}

\vspace{0.2cm}

在图上还可以有路径、环、社区等结构的定义,这些都是由节点、边组成的更高层次的结构。从最底层、最基本的角度来看,图的基本单元就是节点与边。

要给图赋予动态性,本质上来说就是对图结构的修改。而修改一词具体的含义就包括增加、删除、修改三个方面,具体来看也就是包含以下几种:

\begin{itemize}
    \item 节点的增加、删除
    \item 边的增加、删除
    \item 节点和边属性的变化(增加、删除、修改)
\end{itemize}

\vspace{0.2cm}

其中最关键的在于节点和边的增加部分,对于一部分图而言可能不会有节点与边的删除,但这种删除操作对于某些特定类型的图而言也是经常见到的一种操作。

例如电商平台中用户收藏商品的关系可以看成一个图,上述几种动态性在这个例子中具体的体现形式如下:

\begin{itemize}
    \item 节点的增加——新用户的进入、新商品的上线
    \item 节点的删除——用户的注销、商品的下线(商家主动下线或遭投诉被处理下线)
    \item 边的增加——用户新收藏一个商品
    \item 边的删除——用户对某个商品取消收藏、商品下线导致对该商品的所有收藏关系失效
    \item 节点和边属性的变化——用户修改个人信息、商品修改详情信息、用户修改商品收藏所在的收藏夹
\end{itemize}

\vspace{0.2cm}

以上定义的确可以表述一个图的动态性包含的内容,以节点和边为粒度进行的动态性分析对于基于行为模拟的生成方法比较有效,可以考虑每一种行为对应的概率进行模拟生成。但是这样的方法具有一定的局限性,需要大量的数据对每个节点的行为进行建模,可能会造成效率方面的不足。

\section{图的结构特征与动态性}
\label{cha:structure_dynamic}

上一节从微观的层面进行了图动态性的讨论,完备地说明了图的动态性对于节点和边这些基本结构的影响。但仅止于此并不足以指导生成过程,正如\ref{content:problem}节所说,本文旨在将动态特性体现于基于概率的邻接关系生成过程中,因此需要对动态图中更高层次、更抽象化的特征进行抽取。

\subsection{图的结构特征}

对于真实社交网络图而言,目前已经有许多对其特征进行抽取、总结的工作,学术界也已经有许多公认的社交网络图特征。如:无尺度网络\cite{onnela2007structure}、小世界原理\cite{watts1998collective}、社区结构\cite{girvan2002community}等,这些特征的解释如下:

\begin{itemize}
    \item 无尺度网络(Scale-Free Network):大部分节点的度数很小,只有小部分节点拥有极大的度数;这一特点在很多情况下可以用幂律分布进行建模:
    \begin{equation}
    \label{equ:powerlaw}
    P(k) \sim k^{-\gamma}
    \end{equation}
    这里的$k$表示节点度数,一般情况下$2<\gamma<3$。
    目前有已经发现了满足无尺度网络特性的很多真实网络,如银行间支付网络\cite{Interbank}、语义网络\cite{Steyvers2010The}等。
    \item 小世界原理:真实社交网络中,往往两个节点之间可以通过少数节点联通。这一原理曾经用六度分离理论(Six Degrees of Separation)来表示,也就是说“任意两个人都可以通过最多五个人相认识”。这一概念充分地说明了现实世界社交网络结构的联通性,而一个纯随机的网络是不具有这种特征的。
    在1961年就已经有学者对此进行了实证研究\cite{Gurevitch1961The},哥伦比亚大学也曾经使用电子邮件记录对这一原理进行验证\cite{Dodds827}。
    \item 社区结构:在真实社交网络中,某些节点之间的连接会相对更加紧密,如在同一个小区、同一所学校、具有相同兴趣爱好的人往往有更多的交互。基于此,整个社交网络可以被划分为几个部分,每一部分就是一个社区。社区本质上来说就是一些节点组成的集合(可能会重叠\cite{palla2005uncovering}),其内部的联系比较紧密,社区间的联系更加疏松。具体到图的结构上,也就是同一社区内节点之间连接概率更高、不同社区节点之间连接概率更低。例如:引文网络会按照研究主题形成社区\cite{Michelle2002Girvan}。
\end{itemize}

\vspace{0.2cm}

利用上述特征而非直接着眼于每一个节点、边的行为模拟,可以更高效地进行图结构的生成。

\subsection{基于结构的图生成}
\label{cap:simplegen}

图的生成过程其实就是节点和边的生成,总体上可以划分为结构和属性两部分。本节中主要着眼于图的结构信息。

基于之前对于节点度数分布、社区特征的研究,在忽略属性信息的情况下,可以提出一个简单的有向图生成模型,其中包括如下信息\cite{FastSGG}:

\begin{itemize}
    \item 节点的数量
    \item 边的数量
    \item 边的选择(起点和目标节点分别是哪个已有节点),其中的影响因素包括:
    \begin{itemize}
        \item 每个节点度数分布的拟合函数(包括入度分布与出度分布)
        \item 由于社区结构导致的节点之间连接概率不同
    \end{itemize}
\end{itemize}

\vspace{0.2cm}

其中,社区结构对边生成的影响简单地可以用一个参数$\rho$表示,其含义为社区间生成边概率与社区内部生成边的概率的比值,形式化定义如公式\ref{equ:rho}所示。

\begin{equation}[H]
    \label{equ:rho}
    \rho = \frac{P\left(E_{A,B}\right|\exists C_i, C_j, i\ne j, A\in C_i, B\in C_j)}{P\left(E_{A,B}\right|\exists C_i, A\in C_i, B\in C_i)}
\end{equation}

其中$P\left(E_{A,B}\right)$表示节点$A$和$B$之间有边相连的概率,$C_i, C_j$表示两个不重叠的社区。

将上述生成模型中的参数全部进行定义后,即可进行生成工作。

模型中节点数量、边数量的定义较为简单,下面着重说明模型中节点度数分布、社区分布特征的定义。

基于前人的研究,Power-Law分布(公式\ref{equ:powerlaw})可以用来进行节点度数的拟合。同时每一个社区的构成、社区对边生成影响程度这些社区相关信息也会影响边的生成过程。

除了度数的分布特征的约束,每个节点对应的度数、每个节点度数的相对次序也需要进行约束。这是因为度数分布特征只是能基于概率给出度数为$m$的节点个数,但并没有指定具体度数为$m$的哪些节点是哪些。为了解决这个问题,可以将所有节点进行排序,进而将这些节点分别对应到所有的度数之中。

总体的入度分布与出度分布不一定一致,同一个节点的入度与出度也并不一定相同,甚至每个节点的入度与出度不一定相关。如微博上的网红可能有几百万的关注量,但是他们关注的其他用户个数可能很少;反过来,关注几百万个其他用户的人也并不多。因此,需要将入度分布与出度分布分离,分别进行定义,并且需要分别分析入度分布与出度分布的节点排序特征。

基于对入度分布、出度分布、社区结构的约束,可以生成一个较符合真实网络分布特征的合成图。

\subsection{结构上的动态特征}

上一节中已经抽取出了图结构中的一些特征信息,可以在生成过程中使用。这些特征在图的动态性中会有以下几方面的体现:

\begin{itemize}
    \item 节点的数量变化:新增与删除
    \item 边的数量变化:新增与删除
    \item 边生成过程的变化\footnote{因为生成模型中是基于度数分布和社区分布,因此需要关注与此相关的变化}:
    \begin{itemize}
        \item 分布函数本身的变化(分布函数参数变化、分布函数类型的变化)
        \item 分布中每个节点地位的变化\footnote{节点地位指在所有节点中此节点度数的相对大小,如:原来是A节点度数最大,现在变成B节点度数最大}
        \item 社区分布的变化(社区本身结构的变化、社区参数$\rho$的变化)
    \end{itemize}
\end{itemize}

\vspace{0.2cm}

从真实世界社交网络图的研究来看,分布函数本身的类型发生变化并不常见。幂律分布相对而言能更好拟合真实世界的情况,但是幂律分布中参数$\lambda$在不同网络中有所区别。

\begin{table}[htb]
  \centering
  \caption[事件定义]{各种事件的定义、示例与其所对应的结构动态性}
  \label{tab:events}
  \begin{minipage}[t]{1\textwidth}
    \begin{tabularx}{\linewidth}{lXX}
      \toprule[1.5pt]
      {\heiti 事件} & {\heiti 示例(微博中的用户关注关系)} & {\heiti 结构动态性} \\
      \midrule[1pt]
      节点增长 & 微博的APP在推广后一段时间内有大量用户注册 & 节点增加 \\
      节点删除 & 用户注销账户或被封号 & 节点删除 \\
      边的删除 & 用户取消关注别人 & 边删除 \\
      边的生成 & 用户在浏览过程中关注喜爱的博主 & 边增加 \\
      突发事件$^{*}$ & 新冠肺炎导致医生的关注度增加,疫情之后可能关注度降低 & 分布中节点地位临时变化 \\
      节点重要度变化$^{**}$ & 某大V由于作品受欢迎而逐渐成为网红 & 分布中节点地位变化 \\
      社区参数变化$^{***}$ & 一些有相同爱好(比如都喜欢游戏)的用户互相交流,成为好友,从而自发形成了一个小圈子 & 社区参数$\rho$的变化 \\
      \bottomrule[1.5pt]
    \end{tabularx}\\[2pt]
    \footnotesize *:特指突发事件中对图中节点在所有节点中地位排序的临时性变化,这会导致其度数、度数增长速度受到影响\\ **:特指某些节点在所有节点中地位排序逐渐变化\\ ***:社区参数指公式\ref{equ:rho}中的$\rho$
  \end{minipage}
\end{table}

另外需要注意的是,上述这些变化并不是在所有的社交网络中都会出现。例如,引文网络中一般情况下不会出现边的删除操作。

基于上面的分析,可以将上述各种变化的形式提取出来作为生成时的依据。本节将这些变化的过程抽象为事件,每一种事件都可以描述某一种特定的图结构随时间的变化,如表\ref{tab:events}所示。

事件的定义中需要指定影响的目标、事件发生的时刻。其中一些事件可以具有周期性,如:若一位冰雕手艺人在每个冬季都会发布更多的冰雕相关作品,则会导致其在冬季的关注量增长较快,这就可以认为是季节变化带来的周期性影响。因此,可以将上述各种事件发生的时刻用周期的形式进行定义。

\section{包含动态特征的图生成工具配置}
\label{cha:generatorscheme}

前文中讨论了如何利用更高层、抽象的结构信息对图的生成过程进行分析,提出了一个简单的有向图生成模型,并且提出了用事件的方法进行动态性定义的思路。本节将对\ref{cap:simplegen}节提出的模型进行扩展,加入动态特征,并且用形式化的方式定义其配置方式。

包含动态特征的图生成工具配置需要包括节点、边、社区、事件四大模块,其中动态性主要在事件模块中体现,具体见表\ref{tab:dynamic_define}。

\begin{table}[htb]
  \centering
  \caption[动态图生成工具配置]{包含动态特征的图生成工具配置}
  \label{tab:dynamic_define}
  \begin{minipage}[t]{1\textwidth}
    \begin{tabularx}{\linewidth}{llX}
      \toprule[1.5pt]
      {\heiti 模块} & {\heiti 名称} & {\heiti 说明} \\
      \midrule[1pt]
      节点配置 & 节点标签 & 此类节点的唯一标识符 \\\cline{2-3}
       & 节点个数 & 此类节点在起始时刻的个数,后续节点的增加与删除可以在事件中定义 \\\cline{2-3}
       & 节点属性 & 属性信息配置时须指明属性名、属性类型与取值范围/生成方式 \\\hline
      边配置 & 边标签 & 此类边的唯一标志符 \\\cline{2-3}
       & 边属性 & 属性信息配置时须指明属性名、属性类型与取值范围/生成方式 \\\cline{2-3}
       & 出度分布 & \tabincell{l}{包含:\\-\ \ 分布类型:幂律分布、对数正态分布、均匀分布等$^{*}$\\-\ \ 分布参数:分布函数中需要的参数,如幂律分布中的$\lambda$\\-\ \ 分布的度数范围:最小度数与最大度数} \\\cline{2-3}
       & 入度分布 & 形式上同出度分布,但入度分布只影响每个节点被选择的相对概率大小,不能严格约束节点入度的范围$^{**}$ \\\hline
      社区配置 & 边标签 & 边的唯一标志符$^{***}$ \\\cline{2-3}
       & 比例 & 各个社区的节点数量比例 \\\cline{2-3}
       & 参数 & 表征社区结构紧密度的参数$\rho$ \\\hline
      事件配置 & 事件类型 & 表\ref{tab:events}中定义的事件之一$^{****}$ \\\cline{2-3}
       & 相应标签 & 受事件影响的节点标签/边标签 \\\cline{2-3}
       & 参数 & 事件中需要的参数信息,如:影响范围、影响程度等\\
      \bottomrule[1.5pt]
    \end{tabularx}\\[2pt]
    \footnotesize *:为了能够增加用户配置过程的灵活性,在此不约束必须使用幂律分布\\ **:给定的入度分布会在\ref{cha:DetermineTarget}节中的算法中计算各个节点被选为边的目标节点的概率,但度数范围不能保证(如:共有100个节点、5000条边,但要求入度最大值为3,则此要求不会被满足)\\ ***:由于每种关系(每类边)对应的社区特征可能不同,在此需要说明这个社区特征是在哪类边上的。如:微博上相互关注的人不一定有对方的手机号,因此微博组成的社区与通话记录组成的社区并不相同\\ ****:边的生成无需在事件中进行定义
  \end{minipage}
\end{table}