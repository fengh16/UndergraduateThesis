% !TeX root = ../main.tex

\chapter{引言}
\label{cha:intro}

\section{研究背景}

\subsection{图的概念与应用}

图(Graph)是我们经常会见到的一种数据结构,它由若干给定的节点(Node、Vertex)和节点之间的边(Edge)组成。其中节点可以表示真实世界中的一些事物,而边则可以描述它们之间的一些联系(Relationship)。图的形式化定义为:

\vspace{-8mm}

\begin{equation}
G=(V,E)
\end{equation}

\noindent 其中$V$表示节点的集合,而$E \subseteq\left\{(x, y)|(x, y) \in V^{2}, x \neq y\right\}$表示图中的边(即节点二元组)的集合。

本质上来说,图是一种更优雅地表示多对多关系的一种数据结构,用图的方式进行描述,可以更直观、更有效地展示事物之间的关系,在进行抽象与建模后可以用图上的很多高效的算法来解决实际生活中的问题。

图的历史,可以追溯到欧拉在20世纪80年代对柯尼斯堡七桥问题\cite{biggs1986graph}进行的研究。在这一问题的解决过程中,河岸和岛被抽象成节点,而桥则被抽象成边,进而可以用度数的奇偶性巧妙地解决这一问题,展示了图论的优越性。

我们的生活中很多的关系、事件都可以用图进行建模。如在社会科学的研究过程中,每一个人都可以看作一个节点,而不同人之间的朋友关系、师生关系、都可以看作是节点之间的边。在微博上,不同的用户可以看作是不同的节点,他们之间的好友关系可以看作节点之间的边;而在计算机科学中,通信网络、数据流等许多问题都可以被建模成图论中的问题,一台设备、一个网页都可以被定义为一个节点,从而应用一些较为成熟的图论算法使问题得以解决。

那么,都有哪些问题是用图进行分析得到很好的成果的呢?我们耳熟能详的一笔画问题、最短路径问题等都是可以用图论进行建模、研究的问题,这些问题都有了比较优雅的解决方法。还有四色问题、最大流最小割问题、哈密顿回路问题、社区划分问题等都是在图论范围内的。

\subsection{合成图的必要性}

现在随着互联网的发展,人与人之间的边界被打破,人们之间沟通交流的关系越来越紧密,传递的数据量也越来越大。同时,物联网的蓬勃发展也带来了更多新的需求,随着万物互联时代的到来,越来越多的图数据随着人们的日常生活而产生,刻画着我们愈发丰富的生活,随之而来的是一系列有关信息传播、节点重要性计算的问题,许多学者对此进行了深入的研究。

这些问题与我们的生活息息相关,如:在谣言的传播过程中\cite{daley1965stochastic}\cite{maki1973mathematical}、在文章或广告的转发宣传过程中\cite{Danilevsky2013Information},究竟是哪一些节点起了更大的作用?在互联网上究竟哪一些网站更加权威、更应该在用户搜索结果中被赋予更高的权重\cite{Page98thepagerank}?根据用户对于信息的访问记录,如何能够更好地匹配用户的需求,将更合适的消息推广他/她?如何通过用户之间的好友关系、用户发布的微博等信息来判断出一个用户可能认识的人?如何通过银行账户之间的转账关系,分析得到涉嫌洗钱的可疑账户\cite{pareja2019evolvegcn}?

解决这些问题的算法无论是基于传统方法还是神经网络的方法,都是要使用大量数据进行分析、验证和评估的。最好的解决方法是使用真实世界的数据集进行分析和研究,但是只依赖真实数据的话会有许多问题存在。

一方面,很多真实数据难以收集,收集到的数据可能会人为加入噪声。这一部分是因为一些数据的敏感性,比如银行的转账数据、推特账户的好友关系,可能不会公开或者在公开前会进行一系列混淆、脱敏的操作;并且由于数据属于公司内部资产,对数据的分析可以带来潜在的经济效益,所以很多公司不会选择公开自己内部的数据,并且会对爬取数据的行为进行各种限制。如淘宝的页面就采取了多种反爬虫措施以免自己的数据泄漏。

另外一方面,真实数据集中数据已经固定,通过修改数据的方式得到的新数据不能保证与原数据同样的特征。而且公开的真实数据有限,特别是对于超大规模的社交数据而言更是如此,在这种情况下如果想要找到类似的数据进行对照就不是那么容易。比如,在已经得到推特的数据之后,由于社交网络领域体量较大的公司只有几个,很难找到类似的数据与之进行对比。如果我们对原始数据进行修改(删除或者扩展),使得其中一些特征发生变化,可以解决数据来源少的问题,但是修改的过程本身可能会破坏数据内在的结构,结果可能并不具有参考意义。

因此,合成数据集对于图论相关算法的研究具有很重要的意义,合成数据集的广泛采用可以一定程度上解决数据来源不足的问题。

\section{课题意义}

目前,已经有许多学者对合成数据集的方法进行了探究,其中包括用基于概率模型的传统方法与基于神经网络的方法,两者都各有特色,可以基于已有的真实世界中的图特征先验知识得到较为真实的合成社交网络图。

但目前已有的工作基本上都是生成静态图的,也就是生成的结果只是某个时刻的社交网络图数据,并不具有随时间变化的特征。与之相对应的一个概念就是动态社交网络图(在下文中简称为动态图,对应的英文为Dynamic Social Graph,即随时间流逝会发生变化的社交网络图\cite{sarkar2006dynamic}),相当于若干个静态图组成的时间序列,可以将一个图的演化过程用动态的方式表现出来。与静态图相比,动态图生成相关的研究并不是特别充分。

图的时序变化对于图的分析具有较为重要的意义。真实世界中的社交网络图往往都是动态的,它们是随着时间流逝逐渐建立起来的。随时间的演化信息中蕴涵着某些节点、社区等结构的特征信息以及这些特征的变化趋势,并且可以反映某些事件、现象的影响,如以下例子所示:

\begin{itemize}
  \item 有影响力节点的形成与衰退\cite{braha2006centrality}:在微博的用户关系图中,可以通过某个节点度数增长速度的变化来判断某个大V的形成与过气的过程;
  \item 热点事件的发生:在疫情期间,一些与医疗相关的用户与话题可能会受到更多的关注,相对于其他时间会更加活跃,而这一特征在图结构中的表现就是这些用户与话题的度数增长速度的阶段性提升;
  \item 周期性事件的发生:冰雕手艺人可能在冬天的关注度会更多,与冰激淋相关的商家在夏天往往受到更多的关注,因此长期来看这些节点度数的变化速度会有明显的周期性。
\end{itemize}

\vspace{0.2cm}

如果将动态特性加入到图的生成过程中,让图可以反映类似真实网络中的时序变化,就可以更好地体现社交网络图的演化特征,并且可以让用户更自由地定义符合自己要求的图结构。用控制变量的方法,将某种时序演化结果的图与不同演化条件下的模拟结果相对比,可以更好地为相关算法服务。特别是对于一些动态图的算法而言,可以更自由地生成需要的数据意味着这些算法可以更好地进行设计与验证。

\section{已有研究情况}

目前已经有许多合成图相关的研究,大部分都是合成静态图的,按照原理分类主要包括传统的基于概率的方法和基于神经网络的方法两种。在对已有算法的调研与分析基础上,本节对几种比较典型、有效的合成图数据方法进行介绍。

\subsection{基于传统方法生成}

通过对已有真实社交网络的分析,我们可以得到一些先验知识并构建出符合这些条件的图数据。此部分将对两种比较典型的基于传统的概率模型进行生成的方法进行介绍,分别是S3G2\cite{Minh2012S3G2}和FastSGG\cite{FastSGG}。

S3G2\cite{Minh2012S3G2}这篇工作将节点之间边的形成概率与节点本身的属性建立起了联系,用基于标签相似度的概率算法来进行边的生成,得到了较好的结果。这里提到的属性是指节点本身的一些信息,如人物的年龄、所在城市等,作者观察到相同属性标签的节点之间产生关联的概率相对更大,例如相同时间在同一所大学上学的人之间有好友关系的概率更大。在S3G2的工作中,使用了一个相似度函数将节点转换为一个数字,从而通过排序函数得到不同节点的位置,位置相近的节点有相对较高的相似程度。根据相似程度不同,会给节点分配不同的邻接概率,在此基础上通过随机算法来进行边的生成操作。

在FastSGG\cite{FastSGG}这篇工作中为了能够更高效地进行边的生成操作,同时能够兼顾生成结果的度数分布特征与社区特征,提出了一个通用性的生成算法,并且给出了一个高效的度数分布生成模型,通过用户指定的生成图的一些特征进行针对性的生成。这个算法只需要提供概率密度函数就可以进行计算,对于累计分布函数较难求出的情况也可以很好地解决。而社区结构体现在社区内和社区间生成边的概率不同,通过一个参数可以控制社区结构的明显与否,从而能够得到具有社区结构结果的图结构。这篇工作中使用离散化的方法,将求解累计分布函数逆函数的过程转化为分位点确定的问题,极大地提高了效率,使得FastSGG这篇工作可以胜任超大规模图的生成任务。

\vspace{0.2cm}

\subsection{基于神经网络方法生成}

也有很多学者研究用神经网络的方法进行图的生成,在此对以下几种比较典型的方法进行介绍。

一个图可以用邻接矩阵表示,因此使用GAN或者VAE进行图的邻接矩阵特征的学习,将图的生成问题转化成为一个邻接矩阵的生成问题,而邻接矩阵可以变换为一个大小为$n^2$的向量,从而可以套用生成向量的相关方案。但是这样生成的结果大小固定为$n^2$,比较难进行扩展。同时,由于一个图的邻接矩阵表示与节点排列有关,造成图的向量表示的不唯一性,因此可能需要对所有可能的节点排列都进行分析,或者在所有排列中指定一个具有代表性的排列,这样的操作时间复杂度较高。

使用节点的向量嵌入(Node Embedding)这一方法可以计算出每个节点的向量表示,进而计算节点之间的边生成概率,进行边的生成操作。这种方法能够更精确地衡量节点之间形成边的概率,但也存在一定的缺陷,由于节点的向量嵌入是基于某些给定的图计算出来的,因此这个方法只对给定的节点集合有效,并且只能够在给定一个示例图的基础上进行生成,难以针对给定的多个图进行训练。

也有一些工作基于RNN进行生成,将已经生成的节点与边的相关信息存储在隐藏状态中,进而指导后续生成过程。在GraphRNN \cite{You2018GraphRNN} 的工作中就使用了这种方法,将图生成的过程分解为一系列节点和边的生成,计算新节点与已有节点邻接关系的概率,从而生成社交网络图。GraphRNN可以学习生成与目标集的结构特征相匹配的各种图,存在图和边这两个层次。图层次的RNN可以维护图的状态并且不断产生新的节点,由于不考虑节点属性信息,因此生成新的节点并不需要特别复杂的操作;而在边层次的RNN上,负责为新产生的节点进行边生成操作,边的生成与已生成的节点与边相关。

\subsection{动态图的生成}

据我们所知,目前并没有能够模拟真实世界节点重要性变化、突发事件模拟等时序变化特征的高效生成动态图框架\cite{8573573}\cite{DANCer}。但基本上现有的静态图生成方法都可以进行或者经过修改后进行图生成过程的分步化,也就是将一整个图的生成过程分为几个子过程,每一个过程中只生成部分的节点或者部分的边。本质上来说这样只是将一个图的生成分为几步,生成的图遵循同样的分布,不能体现总体分布特征的变化与一些节点重要程度的变化。

目前有一些基于行为定义的动态图生成方法,如Bob De Caux等人在2013年发表的一篇论文\cite{De2014Dynamic}就是基于代理交互行为的。具体来说,这个方法将网络的生成过程与粒子随机运动相互碰撞的过程相类比,让每一个节点都有对应的移动行为,随机选择其方向与距离以在这一过程中随机遇到其余节点生成对应的边。也有一些算法着重于社交网络图的社区结构,如DANCer\cite{DANCer}这篇工作就是考虑了社区的增长、缩小、合并、节点在社区间的迁移等与社区有关的因素,但其中并没有考虑总体分布的变化以及节点重要程度的变化。

由于动态性也可以用于进行其他类型图的研究,也有一些特定领域上进行图动态生成的研究。在调度算法(如列车时刻表的安排)的研究中,每个时刻的调度都要基于上一时刻的状态,因此可以用动态的方法逐步构建,如Fischer等人的研究\cite{Fischer2011Dynamic}就是在考虑图上最短路径的基础上进行每一时刻的生成任务的。而在知识图谱构建过程中,新的知识在引入时往往与已有内容具有关联,因此知识体系图构建时可以采用动态生成的形式。如图表信息提取、构建知识图的过程中就可以用逐步生成的方式\cite{Kim2017Dynamic},从而建立起来不同概念之间的关联关系。

上述研究中涉及到的图并非社交网络图,其中的生成过程也是人为设计、根据一定规则确定性进行的,与本课题中涉及到的受一些事件影响、由人们的行为生成的动态社交网络图有很大的区别。

\section{本课题要解决的问题}
\label{content:problem}

目前已有的图生成工具存在以下问题:

\begin{enumerate}
  \item 大部分关注于静态图的生成而非动态图的生成,已有的动态图算法无法模拟真实世界中一些事件导致的图结构动态性变化;
  \item 一些已有的用物理模拟的方法\cite{De2014Dynamic}进行动态图生成的工作并不高效;
  \item 使用JSON等配置文件的格式进行配置\cite{FastSGG},用户使用不方便。
\end{enumerate}

\vspace{0.2cm}

为了实现对图动态结构的更好模拟并保证生成的效率,本课题对图结构特征的动态性建模,设计了一个高效的动态图生成算法,此算法可以基于给定的分布信息、事件信息等配置进行动态图的生成。同时为了便于用户使用,避免手写配置文件的繁琐与困难,本课题设计实现了一个完整的动态社交网络图生成管理系统,可以让用户在网页上进行动态图的配置、生成、管理与可视化操作。

本课题主要包含以下几个方面:

\begin{itemize}
  \item 分析真实世界动态社交网络图的特征,定义了几种典型的动态社交网络图事件,模拟真实社交网络图中的动态结构变化;
  \item 在已有的基于给定分布概率进行生成的算法\cite{FastSGG}基础上进行时间序列相关配置的扩展,设计并实现了一个可配置动态社交网络图生成组件;
  \item 完成了一个完整的动态社交网络图生成管理系统,包括前端后端两部分,可以用网页的形式进行动态图的配置、生成与结果的可视化。
\end{itemize}

\section{本文的章节安排}

本文一共6章,每一章的内容安排如下:

第1章是引言,主要介绍图的应用、合成图的必要性、课题意义以及相关工作。

第2章是动态图特征分析与配置,在节点与边、结构特征两个层次上对图的动态性进行了分析,并且提出了一种包含动态特征的图生成配置方式。

第3章是可配置动态图生成组件设计与实现,基于第2章的配置方式,将动态图的生成分解为静态图的生成与事件处理两部分进行介绍,并且介绍了可配置动态图生成组件的整体架构与存储格式等实现细节。

第4章是动态社交网络图生成管理系统,对整个系统的架构、前端设计、执行流程进行了说明。

第5章是实验部分,包括动态图特征分析与性能分析两部分,对生成结果的分布特征、社区特征、事件影响进行了讨论,并且分析了生成用时与图规模、帧数的关系。

第6章是总结,对本文工作的亮点、难点、不足之处进行总结,并对下一步的研究方向进行了展望。