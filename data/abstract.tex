% !TeX root = ../main.tex

% 中英文摘要和关键字

\begin{abstract}
  本论文对动态社交网络图的生成进行了一系列研究,将图结构的时序变化抽象化,使用多种事件进行建模。在此基础上,文中设计实现了一个高效的算法,在给定分布特征、社区结构、事件要求之后可以进行动态图的生成操作,所得结果可以以多种格式进行存储,兼容常用的一些图数据处理与可视化库。文中还设计实现了一个完整的网页端动态图生成管理系统,包括前端后端两部分,将动态图生成工具集成在后端服务中,可以在网页上进行动态图的配置工作,并在网页端完成图生成与结果的可视化任务。

  本文的创新点主要有:
  \begin{itemize}
    \item 使用“事件”这一概念进行动态图中时序变化的定义;
    \item 将一些典型的事件进行建模并抽象化为图结构上的变化;
    \item 设计并完成了一个完整的可配置动态图生成工具;
    \item 设计并完成了一个便于交互操作的动态图生成管理系统,可以便捷地完成动态图的配置、生成、管理与可视化。
  \end{itemize}

  % 关键词用“英文逗号”分隔
  \thusetup{
    keywords = {动态图, 事件, 生成, 管理系统},
  }
\end{abstract}

\begin{abstract*}
  In this paper, I conduct a series of studies on the generation of dynamic social graphs, abstracting the temporal changes of the structure of dynamic graphs and modeling them with a variety of events.On this basis, an efficient algorithm is designed and implemented to generate dynamic graphs using given distribution characteristics, community structure, and event requirements. The results can be stored in a variety of formats, compatible with some commonly used graph data processing and visualization libraries. A complete web-side dynamic graph generation management system is implemented, including front-end and back-end. It can configure the dynamic graph on the web page, and complete the visualization task of graph generation and result on the web page.

  The main innovations of this paper are:
  \begin{itemize}
    \item I use the concept of "events" to define temporal changes in dynamic graphs.
    \item I model some typical events and abstract them into changes in the graph structure.
    \item A complete configurable dynamic graph generator is designed and completed.
    \item A dynamic diagram generation management system is designed and completed, which is easy for interactive operation. It is easy to configure, generate, manage and visualize dynamic diagrams.
  \end{itemize}

  \thusetup{
    keywords* = {dynamic graph, events, generation, management system},
  }
\end{abstract*}
